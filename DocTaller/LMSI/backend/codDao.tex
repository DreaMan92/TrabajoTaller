\definecolor{pblue}{rgb}{0.13,0.13,1}
\definecolor{pgreen}{rgb}{0,0.5,0}
\definecolor{pred}{rgb}{0.9,0,0}
\definecolor{pgrey}{rgb}{0.46,0.45,0.48}


\lstset{language=Java,
backgroundcolor=\color{white},
inputencoding=utf8,
escapeinside={\%*}{*)},
literate={á}{{\'a}}1 {é}{{\'e}}1 {í}{{\'i}}1 {ó}{{\'o}}1 {ú}{{\'u}}1 {Á}{{\'A}}1 {É}{{\'E}}1 {Í}{{\'I}}1 {Ó}{{\'O}}1 {Ú}{{\'U}}1 {ñ}{{\~n}}1 {Ñ}{{\~N}}1,
breakatwhitespace=false,
  showspaces=false,
  showtabs=false,
  breaklines=true,
  showstringspaces=false,
  breakatwhitespace=true,
  commentstyle=\color{pgreen},
  keywordstyle=\color{pblue},
  stringstyle=\color{pred},
  basicstyle=\ttfamily,
  moredelim=[il][\textcolor{pgrey}]{$$},
  moredelim=[is][\textcolor{pgrey}]{\%\%}{\%\%}
}
\begin{lstlisting}[caption=PiezaDao.java (LMSI)]
package com.Taller.WebTaller.dao;

import com.Taller.WebTaller.modelos.Categoria;
import com.Taller.WebTaller.modelos.Marca;
import com.Taller.WebTaller.modelos.Pieza;

import java.util.ArrayList;
import java.util.List;

public interface PiezaDao {

    public List<Pieza> LeerPiezasTodas();
    public List<Pieza> LeerPorMarca(String marca);
    public List<Pieza> LeerPorCategoria(String categoria);
    public List<Marca> ListaMarcas();
    public List<Categoria> ListaCategorias();
    public Pieza piezaPorId(int id);
}
\end{lstlisting}
\clearpage
\begin{lstlisting}[caption=PiezaDaoTodos.java (LMSI)]
package com.Taller.WebTaller.dao;


import com.Taller.WebTaller.modelos.Categoria;
import com.Taller.WebTaller.modelos.Marca;
import com.Taller.WebTaller.modelos.Pieza;
import org.springframework.stereotype.Repository;
import org.springframework.beans.factory.annotation.Autowired;
import org.springframework.jdbc.core.BeanPropertyRowMapper;
import org.springframework.jdbc.core.JdbcTemplate;

import java.util.List;

@Repository
public class PiezaDaoTodos implements PiezaDao{

    @Autowired
    private JdbcTemplate jdbcTemplate;
    @Override
    public List<Pieza> LeerPiezasTodas() {
        String sql = "SELECT id_pieza, marca, modelo, precio, descripcion, stock,url, categorias.nombre as categoria FROM piezas inner join categorias on piezas.id_categoria = categorias.id_categoria";

        return  jdbcTemplate.query(sql,
                BeanPropertyRowMapper.newInstance(Pieza.class));
    }

    @Override
    public List<Pieza> LeerPorMarca(String marca) {
        String sql = "SELECT id_pieza, marca, modelo, precio, descripcion, stock, url, categorias.nombre as categoria FROM piezas inner join categorias on piezas.id_categoria = categorias.id_categoria where marca='"+marca+"'";

        return  jdbcTemplate.query(sql,
                BeanPropertyRowMapper.newInstance(Pieza.class));
    }

    @Override
    public List<Pieza> LeerPorCategoria(String categoria) {
        String sql = "SELECT id_pieza, marca, modelo, precio, descripcion, stock, url, categorias.nombre as categoria FROM piezas inner join categorias on piezas.id_categoria = categorias.id_categoria where categorias.nombre='"+categoria+"'";
        return jdbcTemplate.query(sql,
                BeanPropertyRowMapper.newInstance(Pieza.class));
    }

    @Override
    public List<Marca> ListaMarcas(){
        String sql = "select marca, count(*) as cantidad from PIEZAS group by marca";
        return jdbcTemplate.query(sql,
                BeanPropertyRowMapper.newInstance(Marca.class));
    }
    @Override
    public List<Categoria> ListaCategorias(){
        String sql = "select nombre from categorias";
        return jdbcTemplate.query(sql,
                BeanPropertyRowMapper.newInstance(Categoria.class));
    }

    @Override
    public Pieza piezaPorId(int id) {
        String sql = "select id_pieza,marca,modelo,precio,descripcion,stock,url from piezas where id_pieza="+id;
        return  jdbcTemplate.queryForObject(sql,
                BeanPropertyRowMapper.newInstance(Pieza.class));
    }
}
\end{lstlisting}
\clearpage