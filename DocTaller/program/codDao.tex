\definecolor{pblue}{rgb}{0.13,0.13,1}
\definecolor{pgreen}{rgb}{0,0.5,0}
\definecolor{pred}{rgb}{0.9,0,0}
\definecolor{pgrey}{rgb}{0.46,0.45,0.48}


\lstset{language=Java,
backgroundcolor=\color{white},
inputencoding=utf8,
escapeinside={\%*}{*)},
literate={á}{{\'a}}1 {é}{{\'e}}1 {í}{{\'i}}1 {ó}{{\'o}}1 {ú}{{\'u}}1 {Á}{{\'A}}1 {É}{{\'E}}1 {Í}{{\'I}}1 {Ó}{{\'O}}1 {Ú}{{\'U}}1 {ñ}{{\~n}}1 {Ñ}{{\~N}}1,
breakatwhitespace=false,
  showspaces=false,
  showtabs=false,
  breaklines=true,
  showstringspaces=false,
  breakatwhitespace=true,
  commentstyle=\color{pgreen},
  keywordstyle=\color{pblue},
  stringstyle=\color{pred},
  basicstyle=\ttfamily,
  moredelim=[il][\textcolor{pgrey}]{$$},
  moredelim=[is][\textcolor{pgrey}]{\%\%}{\%\%}
}
\begin{lstlisting}[caption=ConexionDB.java (App Escritorio)]
package DAO;

import java.sql.Connection;
import java.sql.DriverManager;
import java.sql.SQLException;
public class ConexionDB {
	
    public String driver = "oracle.jdbc.driver.OracleDriver";
    public String hostname_port_db = "192.168.100.210:1521:xe";

    // Ruta de nuestra base de datos (desactivamos el uso de SSL con "?useSSL=false")
    public String url = "jdbc:oracle:thin:@" + hostname_port_db ;


    public Connection conectarOracle(String username,String password) {
        Connection conn = null;

        try {
            Class.forName(driver);
            conn = DriverManager.getConnection(url, username, password);
        } catch (ClassNotFoundException | SQLException e) {
            e.printStackTrace();
        }

        return conn;
    }

}
\end{lstlisting}
\clearpage
\begin{lstlisting}[caption=Consultas.java (App Escritorio)]
package DAO;

import java.sql.Connection;
import java.sql.ResultSet;
import java.sql.SQLException;
import java.sql.Statement;
import java.util.ArrayList;
import Modelos.Cliente;
import Modelos.Factura;
import Modelos.Pieza;
import Modelos.Reparacion;
import Modelos.Vehiculo;

public class Consultas {
	
	private String username;
	private String password;
	
	public Consultas(String username,String password) {
		this.username=username;
		this.password=password;
	}
	
	/*Devuelve la ID de la ultima reparacion realizada*/
	public int last_rep_id() throws SQLException {
		ConexionDB SQL = new ConexionDB();
		Connection conn = SQL.conectarOracle(username, password);
		try {
			String sql_id_rep = "select max(id_reparacion) from reparacion";
			Statement pstm = conn.createStatement();
			ResultSet resultado = pstm.executeQuery(sql_id_rep);
			if (resultado.next()) return resultado.getInt(1);
			return 0;
		} catch(Exception e) {
			conn.close();
			return 0;
		}
	}
	
	/*Devuelve la ID de la ultima factura*/
	public int last_factura_id() throws SQLException {
		ConexionDB SQL = new ConexionDB();
		Connection conn = SQL.conectarOracle(username, password);
		try {
			String sql_id_rep = "select max(id_factura) from factura";
			Statement pstm = conn.createStatement();
			ResultSet resultado = pstm.executeQuery(sql_id_rep);
			if (resultado.next()) return resultado.getInt(1);
			return 0;
		} catch(Exception e) {
			conn.close();
			return 0;
		}
	}
	
	/*Devuelve los vehiculos de la base de datos*/
	public ArrayList<Vehiculo> verVehiculos() throws SQLException{
		ArrayList<Vehiculo> vehiculos = new ArrayList<>();
		ConexionDB SQL = new ConexionDB();
		Connection conn = SQL.conectarOracle(username, password);
		try {
			
			String query = "select id_vehiculo,modelo,color, a\u00f1o, matricula,cliente.id_cliente, nombre,apellido,telefono, direccion,dni,correo from vehiculo inner join cliente on vehiculo.id_cliente = cliente.id_cliente";
			Statement pstm = conn.createStatement();
			ResultSet resultados =pstm.executeQuery(query);
			while(resultados.next()) {
				Cliente dueno = new Cliente(resultados.getInt(6), resultados.getString(7), resultados.getString(8), resultados.getString(9), resultados.getString(10), resultados.getString(11), resultados.getString(12) );
				Vehiculo buga = new Vehiculo(resultados.getInt(1), resultados.getString(2), resultados.getString(3),resultados.getDate(4), resultados.getString(5),dueno);
				vehiculos.add(buga);
			}
			conn.close();
			return vehiculos;
		} catch(Exception e) {
			conn.close();
			return vehiculos;
		}
	}
	

	/*Devuelve las piezas de la base de datos*/
	public ArrayList<Pieza> verPiezas() throws SQLException{
		ArrayList<Pieza> piezas = new ArrayList<Pieza>();
		ConexionDB SQL = new ConexionDB();
		Connection conn = SQL.conectarOracle(username, password);

			try {
				String query = "select id_pieza, marca, modelo, precio, stock, descripcion, categorias.nombre from piezas inner join categorias on piezas.id_categoria=categorias.id_categoria";
				Statement pstm =  conn.createStatement();
				ResultSet resultados =pstm.executeQuery(query);
				while(resultados.next()) {
					Pieza pieza = new Pieza(resultados.getInt(1), resultados.getString(2), resultados.getString(3), resultados.getDouble(4), resultados.getInt(5), resultados.getString(6), resultados.getString(7));
					piezas.add(pieza);
				}
				conn.close();
				return piezas;
			}catch (Exception e) {
				conn.close();
				return piezas;
			}

	}
	
	/*Devuelve las facturas de la base de datos*/
	public ArrayList<Factura> verFacturas() throws SQLException{
		ArrayList<Factura> facturas = new ArrayList<Factura>();
		ConexionDB SQL = new ConexionDB();
		Connection conn = SQL.conectarOracle(username, password);
		try {
			String query = "select id_factura, fecha_entrada, precio_total, fecha_fin, pagado, id_vehiculo from factura";
			Statement pstm =  conn.createStatement();
			ResultSet resultados =pstm.executeQuery(query);
			while(resultados.next()) {
				
				Factura fact = new Factura(resultados.getInt(1), resultados.getDate(2), resultados.getDouble(3), resultados.getDate(4), resultados.getBoolean(5), resultados.getInt(6));
				facturas.add(fact);
			}
			conn.close();
			return facturas;
		}catch (Exception e) {
			conn.close();
			return facturas;
		}
	}
	
	/*Devuelve las reparaciones de la base de datos*/
	public ArrayList<Reparacion> verReparaciones() throws SQLException{
		ArrayList<Reparacion> reparaciones = new ArrayList<Reparacion>();
		ConexionDB SQL = new ConexionDB();
		Connection conn = SQL.conectarOracle(username, password);
		try {
			String query = "select * from reparacion";
			Statement pstm =  conn.createStatement();
			ResultSet resultados =pstm.executeQuery(query);
			while(resultados.next()) {
				Reparacion rep = new Reparacion(resultados.getInt(1), resultados.getDate(2), resultados.getFloat(3), resultados.getString(4), resultados.getInt(5));
				reparaciones.add(rep);
			}
			conn.close();
			return reparaciones;
		}catch (Exception e) {
			conn.close();
			return reparaciones;
		}
	}
}
\end{lstlisting}
\clearpage
\begin{lstlisting}[caption=Inserts.java (App Escritorio)]
package DAO;

import java.sql.Connection;
import java.sql.PreparedStatement;
import java.sql.SQLException;
import java.util.Map.Entry;

import Modelos.Factura;
import Modelos.Pieza;
import Modelos.Reparacion;

public class Inserts {
	
	private String username;
	private String password;
	
	public Inserts(String username,String password) {
		this.username=username;
		this.password=password;
	}
	
	/*Inserta una nueva reparacion en la base de datos, junto a las lineas correspondientes en la tabla usa*/
	public void insertarReparacion(Reparacion rep) throws SQLException {
		
		ConexionDB conexion = new ConexionDB();
		Connection conn = conexion.conectarOracle(username, password);
		try {
			String ssql = "insert into reparacion(tiempo_hora,horas_duracion,comentarios, id_factura) values(?,?,?,?)";
			PreparedStatement stmt = conn.prepareStatement(ssql);
			stmt.setDate(1, new java.sql.Date(rep.getHora().getTime()));
			stmt.setFloat(2, rep.getDuracion());
			stmt.setString(3, rep.getComentarios());
			stmt.setInt(4,rep.getId_factura());
			stmt.executeUpdate();
			int id_rep = new Consultas(username,password).last_rep_id();
			for(Entry<Pieza, Integer> entrada : rep.getPiezas().entrySet()) {
				String ssql2 = "insert into usa values(?,"+id_rep+",?)";
				PreparedStatement stmt2 = conn.prepareStatement(ssql2);
				stmt2.setInt(1, entrada.getKey().getId());
				stmt2.setInt(2, entrada.getValue());
				stmt2.executeUpdate();
			}
			conn.close();
		} catch(Exception e) {
			conn.close();
		}
	}
	
	/*Inserta una nueva factura en la bbdd*/
	public void insertarFactura(Factura factura) throws SQLException{
		ConexionDB conexion = new ConexionDB();
		Connection conn = conexion.conectarOracle(username, password);
		try {
			String ssql = "insert into factura(fecha_entrada,precio_total, pagado,id_vehiculo) values(?,?,?,?)";
			PreparedStatement stmt = conn.prepareStatement(ssql);			
			stmt.setDate(1, new java.sql.Date( factura.getFecha_entrada().getTime() ));
			stmt.setDouble(2, factura.getPrecio_total());
			stmt.setInt(3, factura.isPagado() ? 1 : 0);
			stmt.setInt(4,factura.getId_vehiculo());
			stmt.executeUpdate();
			conn.close();	
		} catch(Exception e) {
			e.printStackTrace();
			conn.close();
		}
	}
}
\end{lstlisting}
\clearpage
\begin{lstlisting}[caption=Updates.java (App Escritorio)]
package DAO;

import java.sql.Connection;
import java.sql.PreparedStatement;
import java.sql.SQLException;

import Modelos.Factura;
import Modelos.Pieza;

public class Updates {
	
	private String username;
	private String password;
	
	public Updates(String username,String password) {
		this.username=username;
		this.password=password;
	}
	
	/*Actualiza la factura que ha sido finalizada, estableciendo el precio final y la fecha final*/
	public void actualizarFactura(Factura factura) throws SQLException {
		ConexionDB conexion = new ConexionDB();
		Connection conn = conexion.conectarOracle(username, password);
		try {
			String ssql = "update factura set  precio_total=? , fecha_fin = ?,pagado=? where id_factura=?";
			PreparedStatement stmt = conn.prepareStatement(ssql);
			stmt.setDate(2, new java.sql.Date (factura.getFecha_fin().getTime()));
			stmt.setDouble(1, factura.getPrecio_total());
			stmt.setBoolean(3, factura.isPagado());
			stmt.setInt(4, factura.getId());
			stmt.executeUpdate();
			conn.close();
		} catch(Exception e) {
			conn.close();
		}
	}
	
	/*Cambia el stock de la pieza especificada*/
	public void restarStockPieza(Pieza pieza,int cantidad) throws SQLException {
		ConexionDB conexion = new ConexionDB();
		Connection conn = conexion.conectarOracle(username, password);
		try {
			String ssql = "update piezas set stock=? where id_pieza=?";
			PreparedStatement stmt = conn.prepareStatement(ssql);
			stmt.setInt(1,pieza.getStock()-cantidad);
			stmt.setInt(2, pieza.getId());
			stmt.executeUpdate();
			conn.close();
		} catch(Exception e) {
			conn.close();
		}
	}
}
\end{lstlisting}
\clearpage