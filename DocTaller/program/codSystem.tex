\definecolor{pblue}{rgb}{0.13,0.13,1}
\definecolor{pgreen}{rgb}{0,0.5,0}
\definecolor{pred}{rgb}{0.9,0,0}
\definecolor{pgrey}{rgb}{0.46,0.45,0.48}


\lstset{language=Java,
backgroundcolor=\color{white},
inputencoding=utf8,
escapeinside={\%*}{*)},
literate={á}{{\'a}}1 {é}{{\'e}}1 {í}{{\'i}}1 {ó}{{\'o}}1 {ú}{{\'u}}1 {Á}{{\'A}}1 {É}{{\'E}}1 {Í}{{\'I}}1 {Ó}{{\'O}}1 {Ú}{{\'U}}1 {ñ}{{\~n}}1 {Ñ}{{\~N}}1,
basicstyle=\footnotesize,
breakatwhitespace=false,
  showspaces=false,
  showtabs=false,
  breaklines=true,
  showstringspaces=false,
  breakatwhitespace=true,
  commentstyle=\color{pgreen},
  keywordstyle=\color{pblue},
  stringstyle=\color{pred},
  basicstyle=\ttfamily,
  moredelim=[il][\textcolor{pgrey}]{$$},
  moredelim=[is][\textcolor{pgrey}]{\%\%}{\%\%}
}

\begin{lstlisting}[caption=GestorOperaciones.java (App Escritorio)]
    package System;

    import java.sql.SQLException;
    import java.util.ArrayList;
    import java.util.Date;
    import java.util.Map.Entry;
    
    import DAO.Consultas;
    import DAO.Inserts;
    import DAO.Updates;
    import Modelos.Factura;
    import Modelos.Pieza;
    import Modelos.Reparacion;
    import Modelos.Vehiculo;
    public class GestorOperaciones {
        
        private String username;
        private String password;
        private ArrayList<Vehiculo> coches;
        private ArrayList<Factura> facturas;
        private ArrayList<Pieza> piezas;
        private ArrayList<Reparacion> reparaciones; 
        private Consultas consultasDB;
        private Inserts insertsDB;
        private Updates updatesDB;
        
        public GestorOperaciones(String username, String password) throws SQLException {
            this.username=username;
            this.password=password;
            consultasDB = new Consultas(username,password);
            insertsDB = new Inserts(username,password);
            updatesDB = new Updates(username,password);
            /*Obtenemos los datos de la base de datos*/
            coches = consultasDB.verVehiculos();
            piezas = consultasDB.verPiezas();
            facturas = consultasDB.verFacturas();
            reparaciones = consultasDB.verReparaciones();
            /*Asignamos el vehiculo a cada factura usando las claves primarias*/
            facturas.forEach(factura ->{
                factura.setCoche( buscarPorID( factura.getId_vehiculo() ));
            });
            /*Asignamos a que factura corresponde cada reparacion.*/
            reparaciones.forEach(rep ->{
                Factura factura = buscarFacturaPorId(rep.getId_factura());
                factura.getReparaciones().add(rep);
            });
        }
        
        /*Devuelve la factura con el ID especificado*/
        private Factura buscarFacturaPorId(int id) {
            for(Factura factura : facturas) {
                if (factura.getId()==id) return factura;
            }
            return null;
        }
        
        /*Devuelve el vehiculo con el ID buscado*/
        private Vehiculo buscarPorID(int id) {
            for(Vehiculo coche : coches) {
                if (coche.getId()==id) return coche;
            }
            return null;
        }
        
        /*Devuelve una factura abierta del coche. Si no la encuentra le crea una nueva y la escribe en la base de datos*/
        public Factura buscarFacturaActualCoche(Vehiculo coche) throws SQLException {
            for(Factura factura : facturas) {
                if (factura.getCoche()==coche && factura.getFecha_fin()==null) {
                    return factura;
                }
            }
            
            /*En caso de que sea nueva factura, la escribimos en la base de datos y obtenemos su id*/
            Factura factura = new Factura(coche);
            insertsDB.insertarFactura(factura);
            factura.setId(consultasDB.last_factura_id());
            facturas.add(factura);
            return factura;
        }
    
        /*Actualiza el stock de la pieza restando la cantidad introducida*/
        private void restarPiezas(Pieza pieza, int cantidad) throws SQLException {
            updatesDB.restarStockPieza(pieza, cantidad);
            pieza.setStock(pieza.getStock()-cantidad);
        }
        
        public ArrayList<Pieza> piezasEnStock(){
            ArrayList<Pieza> piezerio = new ArrayList<Pieza>();
            piezas.forEach(piez -> {
                if (piez.getStock()>0) piezerio.add(piez);
            });
            return piezerio;
        }
        /*
         Guarda la reparacion en la base de datos, junto a todas la lineas correspondientes de la tabla USA (piezas-cantidad-reparacion)
         Tambien actualiza el stock de piezas de la tabla piezas.
          */
        public void anadirReparacion(Reparacion rep) throws SQLException {
            insertsDB.insertarReparacion(rep);
            for (Entry<Pieza,Integer> linea : rep.getPiezas().entrySet()) {
                restarPiezas(linea.getKey(),linea.getValue());
            }
        }
        
        /*Cuando un vehiculo ha sido totalmente reparado (1 o mas reparaciones)
         Sobre escribe la linea en la tabla factura, actualizando el precio total, la fecha fin y si esta pagado o no. 
         */
        public void terminarReparacionCoche(Factura factura) throws SQLException {
            factura.setFecha_fin(new Date());
            /*Hacer update en lugar de insert*/
            updatesDB.actualizarFactura(factura);
        }
        
        /*Añade piezas a la reparacion vigente*/
        public void anadirPiezasReparacion(Pieza pieza,int cantidad,Reparacion rep) {
            rep.getPiezas().put(pieza, cantidad);
        }
        
        public ArrayList<Vehiculo> vehiculosEnReparacion(){
            ArrayList<Vehiculo> vehiculos = new ArrayList<Vehiculo>();
            facturas.forEach(factura ->{
                if (factura.getFecha_fin()==null) vehiculos.add(factura.getCoche());
            });
            return vehiculos;
        }
    
        public ArrayList<Vehiculo> getCoches() {
            return coches;
        }
    
        public void setCoches(ArrayList<Vehiculo> coches) {
            this.coches = coches;
        }
    
        public ArrayList<Factura> getFacturas() {
            return facturas;
        }
    
        public void setFacturas(ArrayList<Factura> facturas) {
            this.facturas = facturas;
        }
    
        public ArrayList<Pieza> getPiezas() {
            return piezas;
        }
    
        public void setPiezas(ArrayList<Pieza> piezas) {
            this.piezas = piezas;
        }
    
        public ArrayList<Reparacion> getReparaciones() {
            return reparaciones;
        }
    
        public void setReparaciones(ArrayList<Reparacion> reparaciones) {
            this.reparaciones = reparaciones;
        }	
        
        
    }
    
    
\end{lstlisting}
\clearpage

\begin{lstlisting}[caption=CreadorPDF.java (App Escritorio)]
    package System;

    import java.awt.Color;
    import java.io.File;
    import java.io.FileNotFoundException;
    import java.io.FileOutputStream;
    import java.io.IOException;
    import java.net.MalformedURLException;
    import java.net.URL;
    import java.util.Map.Entry;
    
    import com.itextpdf.io.image.ImageData;
    import com.itextpdf.io.image.ImageDataFactory;
    import com.itextpdf.io.image.ImageType;
    import com.itextpdf.kernel.geom.PageSize;
    import com.itextpdf.kernel.pdf.*;
    import com.itextpdf.layout.*;
    import com.itextpdf.layout.borders.Border;
    import com.itextpdf.layout.borders.DashedBorder;
    import com.itextpdf.layout.borders.DoubleBorder;
    import com.itextpdf.layout.borders.SolidBorder;
    import com.itextpdf.layout.element.BlockElement;
    import com.itextpdf.layout.element.Div;
    import com.itextpdf.layout.element.Image;
    import com.itextpdf.layout.element.List;
    import com.itextpdf.layout.element.ListItem;
    import com.itextpdf.layout.element.Paragraph;
    import com.itextpdf.layout.element.Table;
    import com.itextpdf.layout.element.Text;
    import com.itextpdf.layout.property.ListNumberingType;
    
    import Modelos.Pieza;
    import Modelos.Factura;
    import Modelos.Reparacion;
    
    
    
    
    public class CreadorPDF {
        
        
        private static final Border bordeDoble = new DoubleBorder(1.0f);
        private static final Border bordeSolido = new SolidBorder(1.0f);
        private static final Border borde_seccion = new DashedBorder(1.0f);
        
        public static void crearPdfFactura(Factura factura) throws FileNotFoundException {
            String dest = "facturas/Factura "+factura.getId()+".pdf";
            File file = new File(dest);
            file.getParentFile().mkdirs();
    
            //Inicializar escritor pdf
            FileOutputStream fos = new FileOutputStream(dest);
            PdfWriter writer = new PdfWriter(fos);
            
            //Inicializar documento pdf
            PdfDocument pdf = new PdfDocument(writer);
            
            //Inicializar documento
            Document documento = new Document(pdf,PageSize.A4);
            //Escribir en el documento
            Paragraph cabezero = new Paragraph();
    
            try {
                String url = "/Resources/imagenes/llave-inglesa.png";
                ImageData data = ImageDataFactory.create (CreadorPDF.class.getResource( "/Resources/imagenes/llave-inglesa.png" ));
                Image img = new Image(data);
                cabezero.add(img);
            } catch (Exception e1) {
                // TODO Auto-generated catch block
                e1.printStackTrace();
            }
            cabezero.add(new Text("Hermanos Gutierrez S.A").setFontSize(20f));
            documento.add(cabezero);
            documento.add( new Paragraph("Cliente : " + factura.getCoche().getDueno().toString()) .setBorder(borde_seccion));
            documento.add(new Paragraph("Vehiculo: "+ factura.getCoche().toString()) .setBorder(borde_seccion));
            documento.add(new Paragraph("Fecha entrada: "+ factura.getFecha_entrada()) .setBorder(borde_seccion));
            documento.add(new Paragraph("Fecha Fin: "+ factura.getFecha_fin()) .setBorder(borde_seccion));
            documento.add(new Paragraph("Operaciones:"));
            List lista = new List( ListNumberingType.DECIMAL ) .setBorder(bordeSolido);
            for (Reparacion rep : factura.getReparaciones()) {
                ListItem reparacion = new ListItem();
                
                reparacion.add(new Paragraph("Descripcion de operacion: "+ rep.getComentarios()));
                Table piezas = new Table(4);
                piezas.addHeaderCell("Marca");
                piezas.addHeaderCell("Modelo");
                piezas.addHeaderCell("Precio Ud");
                piezas.addHeaderCell("Cantidad");
                try {
                    for (Entry<Pieza,Integer> linea : rep.getPiezas().entrySet()) {
                        piezas.startNewRow();
                        piezas.addCell(linea.getKey().getMarca());
                        piezas.addCell(linea.getKey().getModelo());
                        piezas.addCell( Double.toString( linea.getKey().getPrecio() ));
                        piezas.addCell(Integer.toString(linea.getValue()));
                    }
                    reparacion.add(piezas);
                } catch(Exception e) {
                    //En caso de una operacion sin piezas
                }
                reparacion.add(new Paragraph("Mano de obra: "+ 10*rep.getDuracion()+"\u20ac"));
                lista.add(reparacion);
    
            } 
            documento.add(lista);
            
            documento.add(new Paragraph("Total piezas y mano de obra: "+ factura.getPrecio_total() + "\u20ac") .setBorder(bordeDoble));
            
            documento.close();
            try {
                java.awt.Desktop.getDesktop().edit(file);
            } catch (IOException e) {
                // TODO Auto-generated catch block
                e.printStackTrace();
            }
        }
    }
\end{lstlisting}
\clearpage