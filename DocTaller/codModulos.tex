\paragraph{Cliente}
\begin{lstlisting}
package Modelos;

public class Cliente {
	private int id;
	private String nombre;
	private String apellido;
	private String telefono;
	private String direccion;
	private String dni;
	private String correo;
	
	public Cliente() {};
	
	public Cliente(int id, String nombre, String apellido, String telefono, String direccion, String dni,
			String correo) {
		this.id = id;
		this.nombre = nombre;
		this.apellido = apellido;
		this.telefono = telefono;
		this.direccion = direccion;
		this.dni = dni;
		this.correo = correo;
	}
	public int getId() {
		return id;
	}
	public void setId(int id) {
		this.id = id;
	}
	public String getNombre() {
		return nombre;
	}
	public void setNombre(String nombre) {
		this.nombre = nombre;
	}
	public String getApellido() {
		return apellido;
	}
	public void setApellido(String apellido) {
		this.apellido = apellido;
	}
	public String getTelefono() {
		return telefono;
	}
	public void setTelefono(String telefono) {
		this.telefono = telefono;
	}
	public String getDireccion() {
		return direccion;
	}
	public void setDireccion(String direccion) {
		this.direccion = direccion;
	}
	public String getDni() {
		return dni;
	}
	public void setDni(String dni) {
		this.dni = dni;
	}
	public String getCorreo() {
		return correo;
	}
	public void setCorreo(String correo) {
		this.correo = correo;
	}
	
	@Override
	public String toString() {
		return nombre+" "+ apellido +"\n"+dni + "\n"+telefono;
	}
	
	
}
\end{lstlisting}
\clearpage

\paragraph{Pieza}
\begin{lstlisting}
package Modelos;

public class Pieza {
	private int id;
	private String marca;
	private String modelo;
	private double precio;
	private int stock;
	private String descripcion;
	private String categoria;
	
	
	public Pieza(int id, String marca, String modelo, double precio, int stock, String descripcion, String categoria) {
		this.id = id;
		this.marca = marca;
		this.modelo = modelo;
		this.precio = precio;
		this.stock = stock;
		this.descripcion = descripcion;
		this.categoria = categoria;
	}
	public int getId() {
		return id;
	}
	public void setId(int id) {
		this.id = id;
	}
	public String getMarca() {
		return marca;
	}
	public void setMarca(String marca) {
		this.marca = marca;
	}
	public String getModelo() {
		return modelo;
	}
	public void setModelo(String modelo) {
		this.modelo = modelo;
	}
	public double getPrecio() {
		return precio;
	}
	public void setPrecio(double precio) {
		this.precio = precio;
	}
	public int getStock() {
		return stock;
	}
	public void setStock(int stock) {
		this.stock = stock;
	}
	public String getDescripcion() {
		return descripcion;
	}
	public void setDescripcion(String descripcion) {
		this.descripcion = descripcion;
	}
	public String getCategoria() {
		return categoria;
	}
	public void setCategoria(String categoria) {
		this.categoria = categoria;
	}
	
	@Override
	public String toString(){
		return marca+" "+modelo + ", Stock:"+ stock;
	}
}
\end{lstlisting}
\clearpage

\paragraph{Vehiculo}
\begin{lstlisting}
package Modelos;

import java.util.Date;

public class Vehiculo {
	
	private int id;
	private String modelo;
	private String color;
	private Date anio;
	private String matricula;

	private Cliente dueno;
	
	
	public Vehiculo(int id, String modelo, String color, Date anio, String matricula, Cliente dueno) {
		this.id = id;
		this.modelo = modelo;
		this.color = color;
		this.anio = anio;
		this.matricula = matricula;
		this.dueno = dueno;
	}
	public int getId() {
		return id;
	}
	public void setId(int id) {
		this.id = id;
	}
	public String getModelo() {
		return modelo;
	}
	public void setModelo(String modelo) {
		this.modelo = modelo;
	}
	public String getColor() {
		return color;
	}
	public void setColor(String color) {
		this.color = color;
	}
	public Date getAnio() {
		return anio;
	}
	public void setAnio(Date anio) {
		this.anio = anio;
	}
	public String getMatricula() {
		return matricula;
	}
	public void setMatricula(String matricula) {
		this.matricula = matricula;
	}
	public Cliente getDueno() {
		return dueno;
	}
	public void setDueno(Cliente dueno) {
		this.dueno = dueno;
	}
	
	@Override
	public String toString() {
		return modelo+" : "+color + " : "+matricula +" : "+anio.toString();
	}	
}
\end{lstlisting}
\clearpage

\paragraph{Reparacion}
\begin{lstlisting}
package Modelos;

import java.util.Date;
import java.util.HashMap;

public class Reparacion {
	
	private int id;
	private Date fecha_hora;
	private float duracion;
	private String comentarios;
	private int id_factura;
	private	HashMap<Pieza,Integer> piezas;
	
	public Reparacion() {};
	
	public Reparacion( Date fecha_hora, float duracion, String comentarios, int id_factura, HashMap<Pieza,Integer> piezas) {
		super();
		this.fecha_hora = fecha_hora;
		this.duracion = duracion;
		this.comentarios = comentarios;
		this.id_factura = id_factura;
		this.piezas=piezas;
	}

	public Reparacion(int id, Date fecha_hora, float duracion, String comentarios, int id_factura) {
		super();
		this.id = id;
		this.fecha_hora = fecha_hora;
		this.duracion = duracion;
		this.comentarios = comentarios;
		this.id_factura = id_factura;
	}
	
	public Date getFecha_hora() {
		return fecha_hora;
	}


	public void setFecha_hora(Date fecha_hora) {
		this.fecha_hora = fecha_hora;
	}


	public int getId_factura() {
		return id_factura;
	}


	public void setId_factura(int id_factura) {
		this.id_factura = id_factura;
	}


	public int getId() {
		return id;
	}
	public void setId(int id) {
		this.id = id;
	}
	public Date getHora() {
		return fecha_hora;
	}
	public void setHora(Date hora) {
		this.fecha_hora = hora;
	}
	public float getDuracion() {
		return duracion;
	}
	public void setDuracion(float duracion) {
		this.duracion = duracion;
	}
	public String getComentarios() {
		return comentarios;
	}
	public void setComentarios(String comentarios) {
		this.comentarios = comentarios;
	}

	public HashMap<Pieza, Integer> getPiezas() {
		return piezas;
	}
	public void setPiezas(HashMap<Pieza, Integer> piezas) {
		this.piezas = piezas;
	}	
}
\end{lstlisting}
\clearpage

\paragraph{Factura}
\begin{lstlisting}
package Modelos;

import java.util.ArrayList;
import java.util.Date;
import java.util.Map.Entry;

public class Factura {
	
	private int id=0;
	private Date fecha_entrada;
	private double precio_total;
	private Date fecha_fin = null;
	private boolean pagado = false;
	private int id_vehiculo;
	private Vehiculo coche;
	private ArrayList<Reparacion> reparaciones = new ArrayList<Reparacion>();
	
	public Factura(Vehiculo coche) {
		this.id_vehiculo=coche.getId();
		this.coche=coche;
		fecha_entrada = new Date();
	};
	
	public Factura(int id, Date fecha_entrada, double precio_total, Date fecha_fin, boolean pagado, int id_vehiculo) {
		this.id = id;
		this.fecha_entrada = fecha_entrada;
		this.precio_total = precio_total;
		this.fecha_fin = fecha_fin;
		this.pagado = pagado;
		this.id_vehiculo=id_vehiculo;
	}
	public Factura(int id, Date fecha_entrada, double precio_total, Date fecha_fin, boolean pagado, Vehiculo coche) {
		this.id = id;
		this.fecha_entrada = fecha_entrada;
		this.precio_total = precio_total;
		this.fecha_fin = fecha_fin;
		this.pagado = pagado;
		this.coche = coche;
	}
	
	public ArrayList<Reparacion> getReparaciones() {
		return reparaciones;
	}

	public void setReparaciones(ArrayList<Reparacion> reparaciones) {
		this.reparaciones = reparaciones;
	}

	public int getId() {
		return id;
	}
	public void setId(int id) {
		this.id = id;
	}
	public Date getFecha_entrada() {
		return fecha_entrada;
	}
	public void setFecha_entrada(Date fecha_entrada) {
		this.fecha_entrada = fecha_entrada;
	}
	/*Se calcula en base a las horas de las reparaciones y las piezas utilizadas.
	 Se podria poner que tuviese dependencia de la clase del vehiculo. 
	 */
	public double getPrecio_total() {
		precio_total =0;
		reparaciones.forEach(rep ->{
			precio_total+=rep.getDuracion()*10;
			try {
				for (Entry<Pieza,Integer> linea : rep.getPiezas().entrySet()) {
					precio_total+=linea.getValue()*linea.getKey().getPrecio();
				}
				
			} catch (Exception e){
				//En caso de operacion sin piezas
			}
		});
		return precio_total;
	}
	
	public Date getFecha_fin() {
		return fecha_fin;
	}
	public void setFecha_fin(Date fecha_fin) {
		this.fecha_fin = fecha_fin;
	}
	public boolean isPagado() {
		return pagado;
	}
	public void setPagado(boolean pagado) {
		this.pagado = pagado;
	}
	public Vehiculo getCoche() {
		return coche;
	}
	public void setCoche(Vehiculo coche) {
		this.coche = coche;
	}
	public int getId_vehiculo() {
		return id_vehiculo;
	}
	public void setId_vehiculo(int id_vehiculo) {
		this.id_vehiculo = id_vehiculo;
	}
}
\end{lstlisting}
\clearpage